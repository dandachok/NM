\section*{Теория}

$LU$ – разложение матрицы $A$ представляет собой разложение 
матрицы $A$ в произведение
нижней и верхней треугольных матриц, т.е.
$$A= LU$$
где $L$ - нижняя треугольная матрица, $U$ - верхняя
треугольная матрица.

$LU$ – разложение может быть построено с использованием
метода Гаусса. В результате прямого
хода метода Гаусса получим

$$A=A^{(0)} = M^{-1}_1A^{(1)} = M^{-1}_1M^{-1}_2
\dots M^{-1}_{n-1}A^{(n-1)}$$
Где $A^{(n-1)} = U$ верхняя треугольная матрица,
а $L=M^{-1}_1M^{-1}_2\dots M^{-1}_{n-1}$ нижняя треугольная
матрица, имеющая вид

$$L=\begin{pmatrix}
    1 & 0 & 0 & \ldots & 0 & 0 \\
    \mu^{(1)}_2 & 1 & 0 & \ldots & 0 & 0 \\
    \mu^{(1)}_3 & \mu^{(2)}_3 & 1 & \ldots & 0 & 0 \\
    \vdots & \vdots & \vdots & \ddots & 0 & 0 \\
    \mu^{(1)}_n & \mu^{(2)}_n & \mu^{(2)}_n & \dots & \mu^{(n-1)}_n & 1 \\

\end{pmatrix}$$

Таким образом, искомое разложение $A = LU$ получено.

В дальнейшем $LU$ – разложение может быть эффективно
использовано при решении систем
линейных алгебраических уравнений вида $Ax = b$.
Действительно, подставляя $LU$ – разложение
в СЛАУ, получим $LUx = b$, или $Ux=L^{-1}y$.
Т.е. процесс решения
СЛАУ сводится к двум простым
этапам.

На первом этапе решается СЛАУ $Lz = b$. Поскольку матрица
системы - нижняя треугольная,
решение можно записать в явном виде:

$$z_1=b_1,\ 
z_i = b_i + \sum^i_{j=1} l_{ij}z_j,
\qquad i = \overline{2,n}$$

На втором этапе решается СЛАУ $Ux = z$ с верхней треугольной
матрицей. Здесь, как и на
предыдущем этапе, решение представляется в явном виде:

$$x_n = \frac{z_n}{u_{nm}},\ 
x_i = \frac{1}{u_{ii}}\left(z_i - \sum^n_{j=i+1} u_{ij}x_j\right),
\qquad i = \overline{n-1,1}$$

Отметим, что второй этап эквивалентен обратному ходу
методу Гаусса, тогда как первый
соответствует преобразованию правой части СЛАУ в процессе
прямого хода.
\pagebreak