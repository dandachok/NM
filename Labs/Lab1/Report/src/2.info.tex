\section*{Теория}

Метод прогонки является одним из эффективных методов
решения СЛАУ с трехдиагональными
матрицами, возникающих при конечно-разностной аппроксимации
задач для обыкновенных
дифференциальных уравнений (ОДУ) и уравнений в частных
производных второго порядка и
является частным случаем метода Гаусса. Рассмотрим
следующую СЛАУ:

$$\left\{\begin{aligned}
    b_1x_1 + c_1&x_2 = d_1 \\
    a_2x_1 + b_2&x_2 + c_2x_3 = d_2 \\
    a_3&x_2 + b_3x_3 + c_3x_4 = d_3 \\
    \\
    & a_{n-1}x_{n-2} + b_{n-1}x_{n-1} + c_{n-1}x_{n} = d_{n-1} \\
    & a_{n}x_{n-1} + b_{n}x_{n} = d_{n} \\
\end{aligned}\right.$$

решение которой будем искать в виде

$$x_i = P_ix_{i+1} + Q_i, \qquad i=\overline{1,n}$$

где $P$, $Q$ – прогоночные коэффициенты,
подлежащие определению.

Прогоночные коэффициенты вычисляются по следующим формулам
$$\begin{alignedat}{6}
    &P_1 = -\frac{c_1}{b_1}, &Q_1 &= \frac{d_1}{b_1}, &i = 1\\
    &P_i = -\frac{c_i}{b_i + a_iP_{i-1}},
        \qquad &Q_i &= \frac{d_i - a_iQ_{i-1}}{b_i + a_iP_{i-1}},
        \qquad &i=\overline{2,n-1} \\
    &P_n = 0, &Q_n &= \frac{d_i - a_iQ_{i-1}}{b_i + a_iP_{i-1}},
        &i=n\\
\end{alignedat}$$
Обратный ход метода прогонки осуществляется в соответствии с выражением

Общее число операций в методе прогонки равно
$8n+1$, т.е. пропорционально числу уравнений. Такие методы
решения СЛАУ называют экономичными. Для сравнения число
операций в методе Гаусса пропорционально $n^3$.
Для устойчивости метода прогонки достаточно
выполнение следующих условий

$$a_i \neq 0,\ c_i \neq 0, \qquad i = \overline{2,n-1}$$
$$|b_i| \geqslant |a_i| + |c_i| \qquad i = \overline{1,b}$$

Причем строгое неравенство имеет место хотя бы при одном.
Здесь устойчивость понимается в
смысле не накопления погрешности решения в ходе
вычислительного процесса при малых
погрешностях входных данных (правых частей и элементов
матрицы СЛАУ).
\pagebreak