\section*{Теория}

\subsection*{Метод Ньютона}

При нахождении корня уравнения методом Ньютона,
итерационный процесс определяется
формулой

$$x^{(k+1)} = x^{(k)}-\frac{f(x^{(k)})}{f'(x^{(k)})}$$

Для начала вычислений требуется задание начального приближения

\textbf{Теорема.} Пусть на отрезке $[a, b]$ функция $f(x)$ имеет первую и
вторую производные постоянного знака и пусть $f(a)f(b) < 0$.

Тогда если точка $x^{(0)}$ выбрана на $[a, b]$ так, что 

$$f(x^{(0)})f''(x^{(0)})>0$$

то начиная с неё последовательность $\{x^{(k)}\}$, $(k=0,1,2,\dots)$,
определяемая методом Ньютона монотонно сходится к корню 
$x^* \in [a,b]$ уравения.

В качестве условия окончания итераций в практических вычислениях часто
используется правило

$$|x^{(k+1)}-x^{(k)}| < \varepsilon \Rightarrow x^* \approx x^{(k+1)}$$

\subsection*{Метод простой итерации}

При использовании метода простой итерации уравнение заменяется эквивалентным
уравнением с выделенным линейным членом

$$x = \varphi(x)$$

Решение ищется путем построения последовательности

$$x^{(k+1)} = \varphi(x^{(k)}), \qquad k = 1,2,\dots$$

начиная с некоторого заданного значения $x^{(0)}$. Если $\varphi(x)$ - непрерывная
функция, а $x^{(k+1)} = \varphi(x^{(k)}), \qquad k = 1,2,\dots$ -
сходящаяся последовательность, то значение
$x^* = \lim_{k\rightarrow \infty} x^{(k)}$
является решением уравнения.

\pagebreak

\textbf{Теорема.} Пусть функция $\varphi(x)$ - определена и 
дифференцируема на отрезке $[a,b]$. Тогда, если выполненны условия:
\begin{enumerate}
    \item $\varphi(x)\in[a,b], \forall x\in[a,b]$
    \item $\exists q:|\varphi'(x)|\leq q < 1, \forall x\in (a,b)$
\end{enumerate}

то уравнение имеет единственный корень на $[a, b]$. К этому корню
сходится одределяемая методом простых итераций последовательность
$\{x^{(k)}\}$, $(k=0,1,2,\dots)$, начиная с любого $x^{(0)}\in[a,b]$.

При этом справедливы оценки погрешности $(\forall k \in N)$:

$$|x^* - x^{(k+1)}|\leq\frac{q}{1-q}|x^{(k+1)}-x^{(k)}|$$
$$|x^* - x^{(k+1)}|\leq\frac{q^{k + 1}}{1-q}|x^{(1)}-x^{(0)}|$$


\pagebreak