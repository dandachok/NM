\section*{Теория}

\subsection*{Метод Ньютона}

Если определено начальное приближение 
$\textbf{x}^{(0)}=(x^{(0)}_1,x^{(0)}_2,\dots,x^{(0)}_n)^T$,
итерационный процесс нахождения решения системы методом Ньютона можно
представить в виде:

$$\left\{\begin{aligned}
    &x^{(k + 1)}_1 = x^{(k)}_1 + \Delta x^{(k)}_1 \\
    &x^{(k + 1)}_2 = x^{(k)}_2 + \Delta x^{(k)}_2 \\
    &\dots\dots\dots\dots\dots\dots \\
    &x^{(k + 1)}_n = x^{(k)}_n + \Delta x^{(k)}_n \\
\end{aligned}\right.$$

где значения приращений
$\Delta x^{(k)}_1,\Delta x^{(k)}_2,\dots,\Delta x^{(k)}_n$
определяются из решения системы
линейных алгебраических уравнений, все коэффициенты которой выражаются через
известное предыдущее приближение
$\textbf{x}^{(k)}=(x^{(k)}_1,x^{(k)}_2,\dots,x^{(k)}_n)^T$

$$\left\{\begin{aligned}
    &f_1(\textbf{x}^{(k)})
        + \frac{\partial f_1(\textbf{x}^{(k)})}{\partial x_1}\Delta x^{(k)}_1
        + \frac{\partial f_1(\textbf{x}^{(k)})}{\partial x_2}\Delta x^{(k)}_2 + \dots
        + \frac{\partial f_1(\textbf{x}^{(k)})}{\partial x_n}\Delta x^{(k)}_n = 0 \\
    &f_2(\textbf{x}^{(k)})
        + \frac{\partial f_2(\textbf{x}^{(k)})}{\partial x_1}\Delta x^{(k)}_1
        + \frac{\partial f_2(\textbf{x}^{(k)})}{\partial x_2}\Delta x^{(k)}_2 + \dots
        + \frac{\partial f_2(\textbf{x}^{(k)})}{\partial x_n}\Delta x^{(k)}_n = 0 \\
    &\dots\dots\dots\dots\dots\dots\dots\dots\dots\dots\dots\dots\dots
        \dots\dots\dots\dots\dots\dots\dots\dots\dots\dots \\
    &f_n(\textbf{x}^{(k)})
        + \frac{\partial f_n(\textbf{x}^{(k)})}{\partial x_1}\Delta x^{(k)}_1
        + \frac{\partial f_n(\textbf{x}^{(k)})}{\partial x_2}\Delta x^{(k)}_2 + \dots
        + \frac{\partial f_n(\textbf{x}^{(k)})}{\partial x_n}\Delta x^{(k)}_n = 0 \\
\end{aligned}\right.$$

В векторно-матричной форме расчетные формулы имеют вид


\begin{equation}
    \label{1}
    \textbf{x}^{(k+1)} = \textbf{x}^{(k)} + \Delta \textbf{x}^{(k)}
    \qquad k=1,2,\dots
\end{equation}

где вектор приращений
$\Delta\textbf{x}^{(k)} = \begin{pmatrix}
    x^{(k)} \\
    x^{(k)} \\
    \vdots \\
    x^{(k)} \\
\end{pmatrix}$
находится из решения уравнения

\begin{equation}
    \label{2}
    \textbf{f}(\textbf{x}^{(k)}) + \textbf{J}(\textbf{x}^{(k)})
+ \Delta \textbf{x}^{(k)} = 0\end{equation}

Здесь $\textbf{J}(\textbf{x}) = \begin{bmatrix}
    \cfrac{\partial f_1(\textbf{x})}{\partial x_1} &
    \dfrac{\partial f_1(\textbf{x})}{\partial x_2} & \hdots &
    \cfrac{\partial f_1(\textbf{x})}{\partial x_n} \\
    \cfrac{\partial f_2(\textbf{x})}{\partial x_1} &
    \dfrac{\partial f_2(\textbf{x})}{\partial x_2} & \hdots &
    \cfrac{\partial f_2(\textbf{x})}{\partial x_n} \\
    \vdots & \vdots & \ddots & \vdots \\
    \cfrac{\partial f_n(\textbf{x})}{\partial x_1} &
    \dfrac{\partial f_n(\textbf{x})}{\partial x_2} & \hdots &
    \cfrac{\partial f_n(\textbf{x})}{\partial x_n} \\
\end{bmatrix}$ - матрица Якоби.

Выражая из \eqref{2} вектор приращений $\Delta\textbf{x}^{(k)}$
и подставляя его в \eqref{1},
итерационный процесс нахождения решения можно записать в виде

$$\textbf{x}^{(k+1)}=\textbf{x}^{(k)}-\textbf{J}^{-1}\textbf{f}\left(\textbf{x}^{(k)}\right)$$

В практических вычислениях в качестве условия окончания итераций обычно
используется критерий

$$\left\Vert\textbf{x}^{(k)} - \textbf{x}^{(k)}\right\Vert \leqslant \varepsilon$$

\subsection*{Метод простой итерации}

При использовании метода простой итерации система
уравнений приводится к эквивалентной системе специального вида

 $$\left\{\begin{aligned}
    & x_1 = \varphi_1(x_1, x_2, \dots, x_n) \\ 
    & x_2 = \varphi_2(x_1, x_2, \dots, x_n) \\
    & \dots \dots \dots \dots \dots \dots \dots \\
    & x_3 = \varphi_3(x_1, x_2, \dots, x_n) \\
 \end{aligned}\right.$$

Если выбрано некоторое начальное приближение
$\textbf{x}^{(0)}=(x^{(0)}_1,x^{(0)}_2,\dots,x^{(0)}_n)^T$,
последующие приближения в методе простой итерации находятся по формулам

$$\left\{\begin{aligned}
    & x_1^{(k + 1)} = \varphi_1(x_1^{(k)}, x_2^{(k)}, \dots, x_n^{(k)}) \\ 
    & x_2^{(k + 1)} = \varphi_2(x_1^{(k)}, x_2^{(k)}, \dots, x_n^{(k)}) \\
    & \dots \dots \dots \dots \dots \dots \dots \\
    & x_n^{(k + 1)} = \varphi_n(x_1^{(k)}, x_2^{(k)}, \dots, x_n^{(k)}) \\
 \end{aligned}\right.$$

\textbf{Теорема.} Пусть вектор-функция $\varphi(\textbf{x})$ непрерывна,
вместе со своей производной
$$\varphi(\textbf{x}) = \begin{bmatrix}
    \cfrac{\partial \varphi_1(\textbf{x})}{\partial x_1} &
    \dfrac{\partial \varphi_1(\textbf{x})}{\partial x_2} & \hdots &
    \cfrac{\partial \varphi_1(\textbf{x})}{\partial x_n} \\
    \cfrac{\partial \varphi_2(\textbf{x})}{\partial x_1} &
    \dfrac{\partial \varphi_2(\textbf{x})}{\partial x_2} & \hdots &
    \cfrac{\partial \varphi_2(\textbf{x})}{\partial x_n} \\
    \vdots & \vdots & \ddots & \vdots \\
    \cfrac{\partial \varphi_n(\textbf{x})}{\partial x_1} &
    \dfrac{\partial \varphi_n(\textbf{x})}{\partial x_2} & \hdots &
    \cfrac{\partial \varphi_n(\textbf{x})}{\partial x_n} \\
\end{bmatrix}$$

в ограниченной выпуклой замкнутой области $G$ и

$$\max_{x \in G} \left\|\varphi'(\textbf{x}) \right\| \leqslant q < 1$$

где $q$ - постоянная. Если $\textbf{x}^{(0)} \in G$
и все последовательные приближения

$$\textbf{x}^{(k+1)}=\varphi(\textbf{x}^{(k)}), \qquad k = 1,2,\dots$$

содержатся в $G$,
то процесс итерации сходится к единственному решению уравнения

в области G и справедливы оценки погрешности ($\forall k \in N$):

$$\left\| \textbf{x}^{(*)} - \textbf{x}^{(k+1)}\right\|
\leqslant
\frac{q^{(k+1)}}{1 - q}\left\| \textbf{x}^{(1)} - \textbf{x}^{(0)}\right\|$$

$$\left\| \textbf{x}^{(*)} - \textbf{x}^{(k+1)}\right\|
\leqslant
\frac{q}{1 - q}\left\| \textbf{x}^{(k+1)} - \textbf{x}^{(k)}\right\|$$
\pagebreak