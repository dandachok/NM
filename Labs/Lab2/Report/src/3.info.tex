\section*{Теория}

\subsection*{Метод простых итераций}

При большом числе уравнений прямые методы решения СЛАУ (за исключением метода
прогонки) становятся труднореализуемыми на ЭВМ прежде всего из-за сложности хранения и
обработки матриц большой размерности. В то же время характерной особенностью ряда часто
встречающихся в прикладных задачах СЛАУ является разреженность матриц. Число ненулевых
элементов таких матриц мало по сравнению с их размерностью. Для решения СЛАУ с
разреженными матрицами предпочтительнее использовать итерационные методы. Методы
последовательных приближений, в которых при вычислении последующего приближения
решения используются предыдущие, уже известные приближенные решения, называются
итерационными
Рассмотрим СЛАУ

$$\left\{\begin{aligned}
    &a_{11}x_1 + a_{12}x_2 + \dots + a_{1n}x_n = b_1 \\
    &a_{21}x_1 + a_{22}x_2 + \dots + a_{2n}x_n = b_2 \\
    & \qquad \vdots \qquad \vdots \qquad \ddots \qquad \vdots\\
    &a_{n1}x_1 + a_{n2}x_2 + \dots + a_{nn}x_n = b_n \\
\end{aligned}\right.$$

Приведем СЛАУ к эквивалентному виду

$$\left\{\begin{aligned}
    &x_1 = \beta_1 + \alpha_{11}x_1 + \alpha_{12}x_2 + \dots + \alpha_{1n}x_n\\
    &x_2 = \beta_2 + \alpha_{21}x_1 + \alpha_{22}x_2 + \dots + \alpha_{2n}x_n\\
    &\dots\dots\dots\dots\dots\dots\dots\dots\dots\dots\dots\dots\\
    &x_n = \beta_n + \alpha_{n1}x_1 + \alpha_{n2}x_2 + \dots + \alpha_{nn}x_n\\
\end{aligned}\right.$$

или в векторно-матричной форме

$$x = \beta + \alpha x$$

$$x = \begin{pmatrix}
    x_1 \\
    \vdots \\
    x_n
\end{pmatrix},\quad
\beta = \begin{pmatrix}
    \beta_1 \\
    \vdots \\
    \beta_n
\end{pmatrix},\quad
\alpha \neq \begin{pmatrix}
    \alpha_{11} & \hdots & \alpha_{1n} \\
    \vdots & \ddots & \hdots \\
    \alpha_{n1} & \hdots & \alpha_{nn} \\
\end{pmatrix}\quad$$

Разрешим систему относительно неизвестных при ненулевых диагональных элементах
$a_{ii} \neq 0$, $i = \overline{1, n}$ 
(если какой-либо коэффициент на главной диагонали равен нулю, достаточно
соответствующее уравнение поменять местами с любым другим уравнением).

Получим
следующие выражения для компонентов вектора $\beta$ и матрицы $\alpha$  эквивалентной системы

$$\begin{aligned}
    \beta_i = \frac{b_i}{a_{ii}}, &\qquad i = \overline{1,n}; \\
    \alpha_{ij} = -\frac{a_{ij}}{a_{ii}}, &\qquad i = \overline{1,n}, i \neq j; \\
    \alpha_{ij} = 0, &\qquad i = \overline{1,n}, i = j; \\
\end{aligned}$$

При таком способе приведения исходной СЛАУ к эквивалентному виду метод простых
итераций носит название метода Якоби. Тогда метод простых итераций примет вид

$$\left\{\begin{aligned}
    &x^{(0)} = \beta \\
    &x^{(1)} = \beta + \alpha x^{(0)} \\
    &x^{(2)} = \beta + \alpha x^{(1)} \\
    &\dots\dots\dots\dots \\
    &x^{(k)} = \beta + \alpha x^{(k - 1)} \\
\end{aligned}\right.$$

Видно преимущество итерационных методов по сравнению, например, с рассмотренным
выше методом Гаусса. В вычислительном процессе участвуют только произведения матрицы на
вектор, что позволяет работать только с ненулевыми элементами матрицы, значительно
упрощая процесс хранения и обработки матриц.
Имеет место следующее достаточное условие сходимости метода простых итераций.
Метод простых итераций сходится к единственному решению СЛАУ (а,
следовательно, и к решению исходной СЛАУ) при любом начальном приближении, если
какая-либо норма матрицы эквивалентной системы меньше единицы.
Приведем также необходимое и достаточное условие сходимости метода простых итераций.
Для сходимости итерационного процесса необходимо и достаточно, чтобы спектр
матрицы  эквивалентной системы лежал внутри круга с радиусом, равным единице.

\pagebreak

\subsection*{Метод Зейделя}

Метод Зейделя для известного вектора итерации имеет вид:

$$\left\{\begin{aligned}
    &x^{k+1}_{1}=\beta_1 + \alpha_{11}x^{k}_1 + \alpha_{12}x^{k}_2+ \hdots + \alpha_{n}x^{k}_n \\
    &x^{k+1}_{2}=\beta_2 + \alpha_{21}x^{k+1}_1 + \alpha_{22}x^{k}_2+ \hdots + \alpha_{n}x^{k}_n \\
    &x^{k+1}_{3}=\beta_3 + \alpha_{31}x^{k+1}_1 + \alpha_{32}x^{k+1}_2+ \alpha_{33}x^k_3 + \hdots + \alpha_{3n}x^{k}_n \\
    &\dots\dots\dots\dots\dots\dots\dots\dots\dots\dots\dots\dots\dots\\
    &x^{k+1}_{n}=\beta_n + \alpha_{n1}x^{k+1}_1 + \alpha_{n2}x^{k+1}_2 + \hdots + \alpha_{nn-1}x^{k+1}_{n-1} + \alpha_{nn}x^{k}_n \\
\end{aligned}\right.$$

Из этой системы видно $x^{k+1} = \beta + Bx^{k+1} + Cx^k$, что где $B$ - нижняя треугольная матрица с
диагональными элементами, равными нулю, а $C$ - верхняя треугольная
матрица с диагональными элементами, отличными от нуля $\alpha = B + C$. Следовательно
\pagebreak