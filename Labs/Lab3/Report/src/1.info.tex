\section*{Теория}
Задача интерполяции – найти функцию $F(x)$, принимающую в точках $x_i$
те же значения $y_i$. Тогда, условие интерполяции:

$$F(x_i) = y_i$$

При этом предполагается, что среди значений $x_i$ нет одинаковых.

Точки $x_i$ называют узлами интерполяции.

\subsection*{Многочлен Лагранжа}

При глобальной интерполяции на всем интервале $[a,b]$
строится единый многочлен.
Одной из форм записи интерполяционного многочлена для глобальной интерполяции
является многочлен Лагранжа:

$$L_n(x) = \sum_{i=0}^{n}y_i l_i(x)$$

где $l_i(x)$ базисные многочлены степени $n$:

$$l_i= \prod_{\substack{j=1 \\ j\neq i}}^n \frac{x - x_j}{x_i - x_j}=
\frac{(x-x_0)(x-x_1)\dots(x-x_{i-1})(x-x_{i+1})\dots(x-x_n)}
{(x_i-x_0)(x_i-x_1)\dots(x_i-x_{i-1})(x_i-x_{i+1})\dots(x_i-x_n)}$$

Многочлен $l_i(x_j)$ удовлетворяет условию

$$l_i(x_j) = \begin{cases}
    1,\qquad i = j \\
    0,\qquad i \neq j
\end{cases}$$

Это условие означает, что многочлен равен нулю при каждом $x_j$
кроме $x_i$, то есть $x_0,x_1,\dots, x_{i-1},x_{i+1},\dots,x_n$
– корни этого многочлена. Таким образом, степень
многочлена $L_n(x)$ равна $n$ и при $x \neq x_i$ обращаются в ноль
все слагаемые суммы, кроме слагаемого с номером $i = j$, равного $y_i$.

\subsection*{Многочлен Ньютона}

Другая форма записи интерполяционного многочлена – интерполяционный многочлен
Ньютона с разделенными разностями. Пусть функция $f(x)$ задана с произвольным
шагом, и точки таблицы значений пронумерованы в произвольном порядке.

Разделенные разности нулевого порядка совпадают со значениями функции в узлах.
Разделенные разности первого порядка определяются через разделенные разности
нулевого порядка:

$$f(x_i, x_{i+1})=\frac{f(x_{i+1}) - f(x_{i})}{x_{i+1} - x_i}$$

Разделенные разности второго порядка определяются через разделенные разности
первого порядка:

$$f(x_i, x_{i+1}, x_{i+2})=\frac{f(x_{i+1},x_{x+2}) - f(x_{i},x_{i+1})}
{x_{i+2} - x_i}$$

Разделенные разности $k$-го порядка определяются через разделенные разности
порядка $k-1$:

$$f(x_i, x_{i+1}, \hdots, x_{i+k})= \frac{ f(x_{i+1}, \dots, x_{x+k}) -
f(x_{i}, \dots, x_{i+k-1})}
{x_{i+k} - x_i}$$

Используя понятие разделенной разности интерполяционный многочлен Ньютона
можно записать в следующем виде:

\begin{equation*}\begin{split}
    P_n(x) & = f(x_0) + f(x_0, x_1)\cdot(x-x_0) +
    f(x_0, x_1, x_2)\cdot(x-x_0)\cdot(x-x_1) + {}\\
    & f(x_0, x_1,\dots, x_n)\cdot(x-x_0)\cdot(x-x_1)\cdot\hdots\cdot(x-x_{n-1})
\end{split}
\end{equation*}

За точностью расчета можно следить по убыванию членов суммы. Если функция
достаточно гладкая, то справедливо приближенное равенство

$$f(x) - P_n(x) \approx P_{n+1}(x) - P_n(x)$$

Это приближенное равенство можно использовать для практической оценки
погрешности интерполяции:

$$\varepsilon_n = |P_{n+1}(x) - P_{n}|$$

\pagebreak