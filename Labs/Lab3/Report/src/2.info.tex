\section*{Теория}

Использование одной интерполяционной формулы на большом числе узлов
нецелесообразно. Интерполяционный многочлен может проявить свои колебательные
свойства, его значения между узлами могут сильно отличаться от значений
интерполируемой функции. Одна из возможностей преодоления этого недостатка
заключается в применении сплайн-интерполяции. Суть сплайн-интерполяции
заключается в определении интерполирующей функции по формулам одного типа для
различных непересекающихся промежутков и в стыковке значений функции и её
производных на их границах.

Наиболее широко применяемым является случай, когда между любыми двумя точками
разбиения исходного отрезка строится многочлен n-й степени:

$$S(x)=\sum_{k=0}^n a_{ik}x^k,\qquad
x_{i-1}\leqslant x \leqslant x_{i},\quad i=1,\hdots,n$$

который в узлах интерполяции принимает значения аппроксимируемой функции и
непрерывен вместе со своими $(n - 1)$ производными. Такой кусочно-непрерывный
интерполяционный многочлен называется сплайном. Его коэффициенты находятся из
условий равенства в узлах сетки значений сплайна и приближаемой функции, а также
равенства $n - 1$ производных соответствующих многочленов.
На практике наиболеечасто используется интерполяционный многочлен
третьей степени, который удобно представить, как

$$S(x) = a_i + b_i(x-x_{i-1}) + c_i(x-x_{i-1})^2 + d_i(x-x_{i-1})^3$$

где $x_{i-1}\leqslant x \leqslant x_{i},\quad i=1,\hdots,n$

Для построения кубического сплайна необходимо построить $n$ многочленов третьей
степени, т.е. определить $4n$ неизвестных $a$,$b$,$c$,$d$.
Эти коэффициенты ищутся из условий в узлах сетки.

$$\begin{aligned}
&S(x_{i-1}) = a_{i} = 
    a_{i-1} + b_{i-1}(x_{i-1}-x_{i-2}) + c_{i-1}(x_{i-1}-x_{i-2})^2 +
    d_{i-1}(x_{i-1}-x_{i-2})^3 = f_{i-1}\\
&S'(x_{i-1}) = b_i = b_{i-1} + 2c_{i-1}(x_{i-1}-x_{i-2}) +
    3d_{i-1}(x_{i-1}-x_{i-2})^2 \\
&S''(x) = 2c_i = 2c_{i-1} + 6d_{i-1}(x_{i-1}-x_{i-2}) \\
&S(x_0) = a_1 = f_0 \\
&S''(x_0) = c_1 = 0 \\
&S(x_n) = a_n + b_n(x_n-x_{i-1}) + c_n(x_n-x_{n-1})^2 +
    d_n(x_n-x_{n-1})^3 = f_n \\
&S''(x_n) = c_n + 3d_n(x_n - x_{n - 1}) = 0
\end{aligned}$$

Предполагается, что сплайны имеют нулевую кривизну на концах отрезка. В общем
случае могут быть использованы и другие условия.

Если ввести обозначение $h_i = x_i - x_{i-1}$, и исключить из системы
$a_i$,$b_i$,$d_i$, то можно получить систему из $n - 1$ линейных
алгебраических уравнений относительно
$c_i$ , $i = 2,\hdots, n$ с трехдиагональной матрицей:

$$\begin{aligned}
& 2(h_1 + h_2)c_2 + h_2c_3 =
    3\left(\frac{f_2 - f_1}{h_2} - \frac{f_1-f_0}{h_1}\right) \\
& h_{i-1}c_{i-1} + 2(h_{i-1} + h_i)c_i + h_ic_{i+1} =
3\left(\frac{f_i - f_{i-1}}{h_i} - \frac{f_{i-1}-f_{i-2}}{h_{i-1}}\right)
\qquad i = 3,\hdots,n-1 \\
& h_{n-1}c_{n-1} + 2(h_{n-1} + h_i)c_n + =
3\left(\frac{f_n - f_{n-1}}{h_n} - \frac{f_{n-1}-f_{n-2}}{h_{n-1}}\right)
\end{aligned}$$

Остальные коэффициенты сплайнов могут быть восстановлены по формулам:

\begin{align*}
& a_i = f_{i-1} & i & = 1,\hdots,n \\
& b_i = \frac{f_{i}-f_{i-1}}{h_{i}} - \frac{1}{3}h_i(c_{i+1}+2c_i)
    & i & = 1,\hdots,n-1 & \\
& d_i = \frac{c_{i+1}-c_i}{3h_i} & i & = 1,\hdots,n-1 \\
& c_1 = 0 \\
& b_n = \frac{f_n - f_{n-1}}{h_n} - \frac{2}{3}h_nc_n \\
& d_n = - \frac{c_n}{3h_n}
\end{align*}


\pagebreak