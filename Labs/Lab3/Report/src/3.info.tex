\section*{Теория}

\subsection*{Метод наименших квадратов}
Пусть задана таблично в узлах $x_j$ функция $y_j = f(x_j),\quad j=0,1,\hdots,N$.
При этом значения функции $y_j$ определены с некоторой погрешностью,
также из физических соображений известен вид функции, которой должны
приближенно удовлетворять табличные точки, например: многочлен степени $n$,
у которого неизвестны коэффициенты $a_i$,
$F_n(x)=\displaystyle\sum_{i=0}^n a_ix^i$.
Неизвестные коэффициенты будем находить из условия минимума
$i = 0$квадратичного отклонения многочлена от таблично заданной функции.

$$\Phi=\sum_{j=0}^{N}\left[F_n(x_j) - y_j\right]^2$$

Минимума $\Phi$
можно добиться только за счет изменения коэффициентов
многочлена $F_n(x)$. Необходимые условия экстремума имеют вид

$$\frac{\partial\Phi}{\partial a_k}=2\sum_{j=0}^{N}
\left[\sum_{i=0}^{n} a_ix_j^i - y_j\right]x_j^k=0,\qquad k = 0,1,\hdots,n$$

Эту систему для удобства преобразуют к следующему виду:

$$\sum_{i=0}^{n}a_i\sum_{j=0}^{N}x_j^{k+i} =
\sum_{j=0}^{N} y_jx_j^k,\qquad k = 0,1,\hdots,n$$

Такая система называется нормальной системой метода наименьших квадратов
(МНК) представляет собой систему линейных алгебраических уравнений относительно
коэффициентов $a_i$. Решив систему, построим многочлен $F_n(x)$, приближающий
функцию $f(x)$ и минимизирующий квадратичное отклонение.
\pagebreak