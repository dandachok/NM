\section*{Теория}

Формулы численного дифференцирования в основном используются при нахождении
производных от функции, заданной таблично. Исходная функция, заменяется
некоторой приближающей, легко вычисляемой функцией. Наиболее часто в качестве
приближающей функции берется интерполяционный многочлен, а производные
соответствующих порядков определяются дифференцированием многочлена.
При решении практических задач, как правило, используются аппроксимации первых и
вторых производных.

В первом приближении, таблично заданная функция может быть аппроксимирована
отрезками прямой

$$y(x) \approx \varphi(x) = y_i + \frac{y_{i+1} - y_i}{x_{i+1}-x_i}(x-x_i),
\qquad x \in [x_i,x_{i+1}]$$

Тогда

$$y'(x) \approx \varphi'(x) = \frac{y_{i+1} - y_i}{x_{i+1}-x_i} = const,
\qquad x \in [x_i,x_{i+1}]$$

производная является кусочно-постоянной функцией и рассчитывается с первым
порядком точности в крайних точках интервала, и со вторым порядком точности в
средней точке интервала.

Для вычисления второй производной, необходимо использовать интерполяционный
многочлен, как минимум второй степени. После дифференцирования многочлена
получаем

$$y''(x) \approx \varphi''(x) = 2\cfrac
{\dfrac{y_{i+2} - y_{i+1}}{x_{i+2}-x_{i+1}} - \dfrac{y_{i+1} - y_i}{x_{i+1}-x_i}}
{x_{i+2}-x_i}, \qquad x \in [x_i,x_{i+1}]$$
\pagebreak