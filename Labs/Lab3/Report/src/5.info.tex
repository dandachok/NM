\section*{Теория}

Формулы численного интегрирования используются в тех случаях, когда вычислить
аналитически определенный интеграл не удается. Отрезок разбивают точками с
достаточно мелким шагом и на одном или нескольких отрезках подынтегральную
функцию заменяют такой приближающей так что она, во-первых, близка, а, во-вторых,
интеграл от неё легко вычисляется. В нашем случае будем использовать
интерполяционный многочлен, при чем коэффиценты различны на каждом отрезке.

$$f(x) = P_n(x, \overline{a}_i) + R_n(x, \overline{a}_i),
\qquad x \in [x_i,x_{i+k}]$$

где $R_n$ - остаточный член интерполяции.

Тогда

$$F = \sum_{i=1}^{N}\int\limits_{x_i-1}^{x_i}P_n(x, \overline{a}_i)\,dx + R$$

где $R = \sum_{i=1}^{N}\int\limits_{x_i-1}^{x_i}R_n(x, \overline{a}_i)\,dx$ -
остаточный член формулы численного интегрирования или её погрешность. 

Заменим подынтегральную функцию, интерполяционным многочленом Лагранжа
нулевой степени, проходящим через середину отрезка, получим формулу
прямоугольников.

$$\int\limits_a^b f(x)\,dx \approx
\sum_{i=1}^N h_i f\left(\frac{x_{i-1 - x_i}}{2}\right)$$

В случае таблично заданных функций удобно в качестве узлов интерполяции выбрать
начало и конец отрезка интегрирования, т.е. заменить функцию многочленом Лагранжа
первой степени.

$$F = \int\limits_a^b f(x)\,dx \approx
\frac{1}{2}\sum_{i=1}^N (f_{i-1} + f_i)h_i$$

Эта формула носит название формулы трапеций.

Для повышения порядка точности формулы численного интегрирования заменим
подынтегральную кривую параболой – интерполяционным многочленом второй
степени, выбрав в качестве узлов интерполяции концы и середину отрезка
интегрирования.

Для случая $h_i = \cfrac{x_i - x_{i-1}}{2}$, получим формулу Симпсона(парабол):

$$F = \int\limits_a^b f(x)\,dx \approx
\frac{1}{3}\sum_{i=1}^N (f_{i-1} + 4f_{i-\frac{1}{2}} + f_i)h_i$$

Метод Рунге-Ромберга-Ричардсона позволяет получать более высокий порядок
точности вычисления. Если имеются результаты вычисления определённого
интеграла на сетке с шагом $h - F = F_h +O(h^p)$ и на сетке
с шагом $kh - F = F_{kh} +O((kh)^p)$, то

$$F = \int\limits_a^b f(x)\,dx =
F_h + \frac{F_h - F_{hk}}{k^p - 1} + O(h^{p+1})
$$
\pagebreak