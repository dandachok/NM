\section*{Теория}

В основе $QR$-алгоритма лежит представление матрицы в виде, где $A = QR$ - ортогональная
матрица, $R$ - верхняя треугольная. Такое разложение существует для любой квадратной
матрицы. Одним из возможных подходов к построению $QR$ разложения является
использование преобразования Хаусхолдера, позволяющего обратить в нуль группу
поддиагональных элементов столбца матрицы. Преобразование Хаусхолдера осуществляется с
использованием матрицы Хаусхолдера, имеющей следующий вид:

$$H = E - \frac{2}{\nu^T\nu}\nu\nu^T$$

где $\nu$ - произвольный ненулевой вектор-столбец,
$E$-единичная матрица,
$\nu\nu^T$-квадратная матрицаого же размера

Матрица Хаусхолдера $H_1$ вычисляется

$$\nu^1_1 = a_0 + sign(a^0_{11})\sqrt{\left( \sum^{n}_{j=1}(a^{0}_{j1}) \right)}$$

$$\nu^1_i = a^0_{i1}, \qquad i = \overline{1,n}$$

$$H_1 = E - 2\frac{\nu^1\nu^{1T}}{\nu^{1T}\nu^1}$$

Процедура $QR$ - разложения многократно используется в $QR$ -
алгоритме вычисления собственных значений. Строится следующий итерационный процесс:

$$\begin{aligned}
    &A^{(0)} = A, \\
    &A^{(0)} = Q^{(0)}R^{(0)} - \text{производится QR-разложение}\\
    &A^{(1)} = R^{(0)}Q^{(0)} - \text{перемножение матриц}\\
    &\dots\dots\dots\dots\\
    &A^{(k)} = Q^{(k)}R^{(k)} - \text{производится QR-разложение}\\
    &A^{(k+1)} = R^{(k)}Q^{(k)} - \text{перемножение матриц}\\
\end{aligned}$$

Таким образом, каждому вещественному собственному значению будет соответствовать
столбец со стремящимися к нулю поддиагональными элементами и в качестве критерия
сходимости итерационного процесса для таких собственных значений можно использовать
следующее неравенство:
$$\left( \sum^{n}_{l=m+1}(a^{k}_{lm})^2\right)^{1/2}$$
При этом соответствующее собственное значение принимается
равным диагональному элементу данного столбца.

\pagebreak