\section*{Результаты}

\subsection*{Пример работы программы}

\begin{lstlisting}[language=bash]
(base) MacBook-Air-Dima:Program dandachok$ cat tests/input/part5.t
3

5 -5 -6
-1 -8 -5
2 7 -3

0.1
(base) MacBook-Air-Dima:Program dandachok$ ./part5 <tests/input/part5.t
QR Method iter count: 41
A:
{[-7.358, 5.286, -8.236]
[-7.147, -3.473, 1.266]
[-1.089e-07, 1.565e-08, 4.832]}
(base) MacBook-Air-Dima:Program dandachok$
\end{lstlisting}
В результате получилась матрица:

$$A^{(41)}=
\begin{pmatrix}
    -7.358 & 5.286 & -8.236 \\
    -7.147 & -3.473 & 1.266 \\
    -1.08e-7 & 1.6e-8 & 4.832 \\
\end{pmatrix}$$

Видно, что поддиагональные элементы
достаточно малые, в то же время отчетливо прослеживается
комплексно-сопряженная пара собственных значений, соответствующая
блоку, образуемому элементами первого и
второго столбцов. Несмотря на значительное изменение в ходе
итераций самих этих элементов, собственные значения, соответствующие
данному блоку и определяемые из решения квадратного уравнения
$(a_{11} - \lambda)(a_{22} - \lambda)=a_{12}a_{21}$

После решения уравнения получаются следующие собстенные значения:
$$\lambda_1 \approx -5.4155 + 5.83i,\quad
\lambda_2 \approx -5.4155 - 5.83i,\quad
\lambda_3 \approx 4.832$$

\pagebreak