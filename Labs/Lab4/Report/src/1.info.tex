\section*{Теория}

Рассматривается задача Коши для одного дифференциального уравнения
первого порядка, разрешенного относительно производной

$$\left\{\begin{aligned}
    & y' = f(x) \\
    & y(x_0) = y_0 \\
\end{aligned}\right.$$

Требуется найти решение на отрезке $[a,b]$, где $x_0 = a$

Введем разностную сетку на отрезке $[a, b]$:

$$\Omega = {x_k = x_0 + kh}, \qquad k = 0,1,\dots,N,\quad h = |b-a|/N$$

Формула метода Эйлера:

$$y_{k+1} = y_k + hf(x_k,y_k)$$

Все рассмотренные выше явные методы являются вариантами методов
Рунге-Кутты. Семейство явных методов Рунге-Кутты р-го порядка
записывается в виде совокупности формул:

$$y_{k+1} = y_k + \Delta y$$

$$\Delta y_i = \sum_{j=1}^p c_j K_j^i$$

$$K_i^k = hf\left(x_k + a_ih,
        y_k + h\sum_{j=1}^i b_{ij} K^k_j\right),
        \qquad i = 2,3,\dots,p $$

Метод Рунге-Кутты четвертого порядка точности
$$y_{k+1} = y_k + \Delta y$$

$$\Delta y_k = \frac{1}{6}(K_1^k + 2K_2^k + 2K_3^k + K_4^k)$$

$$\begin{aligned}
    & K_1^k = hf\left(x_k, y_k\right) \\
    & K_2^k = hf\left(x_k + \frac{1}{2}h, y_k + \frac{1}{2}K_1^k\right) \\
    & K_3^k = hf\left(x_k + \frac{1}{2}h, y_k + \frac{1}{2}K_2^k\right) \\
    & K_4^k = hf\left(x_k + h, y_k + K_3^k\right) \\
\end{aligned}$$

Рассматривается задача Коши для системы дифференциальных уравнений
первого порядка разрешенных относительно производной

$$\begin{cases}
    y'_1 = f_1(x, y_1, y_2, \dots, y_n) \\
    y'_2 = f_2(x, y_1, y_2, \dots, y_n) \\
    \dots\dots\dots\dots\dots\dots\dots\hdots \\
    y'_n = f_n(x, y_1, y_2, \dots, y_n) \\
\end{cases}$$

$$\begin{aligned}
    & y_1(x_0) = y_{01} \\ 
    & y_2(x_0) = y_{02} \\ 
    & \dots\dots\dots\dots \\ 
    & y_n(x_0) = y_{0n} \\ 
\end{aligned}$$

Формулы метода Рунге-Кутты 4-го порядка точности для решения системы
следующие

$$y_{k+1} = y_k + \Delta y$$
$$z_{k+1} = z_k + \Delta z$$

$$\Delta y_k = \frac{1}{6}(K_1^k + 2K_2^k + 2K_3^k + K_4^k)$$
$$\Delta z_k = \frac{1}{6}(L_1^k + 2L_2^k + 2L_3^k + L_4^k)$$

$$\begin{aligned}
    & K_1^k = hf\left(x_k, y_k, z_k\right) \\
    & L_1^k = hg\left(x_k, y_k, z_k\right) \\
    & K_2^k = hf\left(x_k + \frac{1}{2}h, y_k + \frac{1}{2}K_1^k, z_k + \frac{1}{2}L_1^k \right) \\
    & L_2^k = hg\left(x_k + \frac{1}{2}h, y_k + \frac{1}{2}K_1^k, z_k + \frac{1}{2}L_1^k \right) \\
    & K_3^k = hf\left(x_k + \frac{1}{2}h, y_k + \frac{1}{2}K_2^k, z_k + \frac{1}{2}L_2^k \right) \\
    & L_3^k = hg\left(x_k + \frac{1}{2}h, y_k + \frac{1}{2}K_2^k, z_k + \frac{1}{2}L_2^k \right) \\
    & K_4^k = hf\left(x_k + h, y_k + K_3^k, z_k + L_3^k\right) \\
    & L_4^k = hg\left(x_k + h, y_k + K_3^k, z_k + L_3^k\right) \\
\end{aligned}$$

При использовании интерполяционного многочлена 3-ей степени построенного
по значениям подынтегральной функции в последних четырех узлах получим
метод Адамса четвертого порядка точности:

$$y_{k+1} = y_k + \frac{h}{24}\left(55f_k - 59f_{k-1}+37f_{k-2}
-9f_{k-3}\right)$$

Метод Адамса как и все многошаговые методы не является самостартующим,
то есть для того, что бы использовать метод Адамса необходимо иметь
решения в первых четырех узлах. В узле решение известно из начальных
условий, а в других трех узлах решения можно получить с помощью
подходящего одношагового метода, например: метода Рунге-Кутты четвертого
порядка.













\pagebreak