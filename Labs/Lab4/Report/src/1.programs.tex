\section*{Реализация}

\subsection*{Метод Эйлера}

\begin{lstlisting}[language=C++]
    class EilerMethod2 {
    public:
        EilerMethod2(Function3* f, double left, double right, double y0, double z0) :
            left(left),
            right(right),
            f(f),
            y0(y0),
            z0(z0) {
        }
    
        void operator() (double h) {
            x = GetValInRange(left, right, h);
            y.clear();
            z.clear();
            y.push_back(y0);
            z.push_back(z0);

            int n = x.size() - 1;
            for (int i = 0; i < n; ++i) {
                double z_next = z[i] + h * (*f)(x[i], y[i], z[i]);
                z.push_back(z_next);
                double y_next = y[i] + h * z[i];
                y.push_back(y_next);
            }
        }

        vdouble GetY() {
            return y;
        }

        vdouble GetZ() {
            return z;
        }
    

    private:
        double left;
        double right;
        Function3* f;
        double h;
        double y0;
        double z0;

        vdouble x;
        vdouble y;
        vdouble z;
};
\end{lstlisting}

\subsection*{Метод Рунге-Кутты}

\begin{lstlisting}[language=C++]
    class RungeKutta2 {
    public:
        RungeKutta2(CauchyProblem t) :
            left(t.left),
            right(t.right),
            f(t.f),
            y0(t.y0),
            z0(t.z0) {
        }
        
        vdouble operator() (double h) {
            x = GetValInRange(left, right, h);
            int n = x.size();
            y.clear();
            z.clear();
            y.push_back(y0);
            z.push_back(z0);

            for (int i = 0; i < n - 1; ++i) {
                vdouble k(ORDER);
                vdouble l(ORDER);
                for (int j = 0; j < ORDER; ++j) {
                    if (j == 0) {
                        l[j] = h * (*f)(x[i], y[i], z[i]);
                        k[j] = h * z[i];
                    }
                    else if (j == 3) {
                        l[j] = h * (*f)(x[i] + h, y[i] + k[j - 1], z[i] + l[j - 1]);
                        k[j] = h * (z[i] + l[j - 1]);
                    } else {
                        l[j] = h * (*f)(x[i] + 0.5*h, y[i] + 0.5*k[j - 1], z[i] + 0.5*l[j - 1]);
                        k[j] = h * (z[i] + 0.5*l[j - 1]);
                    }
                }

                double y_next = y[i] + dif(k);
                double z_next = z[i] + dif(l);
                y.push_back(y_next);
                z.push_back(z_next);
            }

            return y;
        }

        vdouble GetY() {
            return y;
        }

        vdouble GetZ() {
            return z;
        }

        vdouble Get() {
            return y;
        }
        
        Function3* f;
        double left;
        double right;
        double y0;
        double z0;

    private:
        double dif (const vdouble& k) {
            return (k[0] + 2*(k[1] + k[2]) + k[3]) / 6;
        }

        double g (double z) {
            return z;
        }

        static const int ORDER = 4;
        
        vdouble x;
        vdouble y;
        vdouble z;
};
\end{lstlisting}

\subsection*{Метод Адамса}

\begin{lstlisting}[language=C++]
    class AdamsMethod {
        public:
            AdamsMethod(CauchyProblem t) :
                f(t.f),
                left(t.left),
                right(t.right),
                y0(t.y0),
                z0(t.z0),
                task(t) {}
            
            void operator() (double h) {
                RungeKutta2 rk(task);
                rk(h);
                x = GetValInRange(left, right, h);
                y = rk.GetY();
                z = rk.GetZ();
    
                y.resize(4);
                z.resize(4);
    
                int n = x.size();
                for (int i = 3; i < n - 1; ++i) {
                    double z_next = z[i] + h / 24 * df(i);
                    z.push_back(z_next);
                    double y_next = y[i] + h / 24 * dg(i);
                    y.push_back(y_next);
                }
            }
    
            vdouble GetY() {
                return y;
            }
    
            vdouble GetZ() {
                return z;
            }
        private:
    
            double df(int i) {
                double f0 = (*f)(x[i], y[i], z[i]);
                double f1 = (*f)(x[i - 1], y[i - 1], z[i - 1]);
                double f2 = (*f)(x[i - 2], y[i - 2], z[i - 2]);
                double f3 = (*f)(x[i - 3], y[i - 3], z[i - 3]);
    
                return (55*f0 - 59*f1 + 37*f2 - 9*f3);
            }
    
            double dg(int i) {
                return (55*z[i] - 59*z[i - 1] + 37*z[i - 2] - 9*z[i - 3]);
            }
    
            CauchyProblem task;
            Function3* f;
            double left;
            double right;
            double y0;
            double z0;
    
            vdouble x;
            vdouble y;
            vdouble z;
    };
\end{lstlisting}



\pagebreak