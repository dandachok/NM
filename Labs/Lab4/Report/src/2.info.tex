\section*{Теория}

\subsection*{Метод стрельбы}

Суть метода заключена в многократном решении задачи Коши для приближенного
нахождения решения краевой задачи.

Пусть надо решить краевую задачу на отрезке. Вместо исходной
задачи формулируется задача Коши с начальными условиями

$$\begin{aligned}
    & y(a) = \eta \\
    & y'(b) = y_0 \\
\end{aligned}$$

Задачу можно сформулировать таким образом: требуется найти такое значение
переменной $\eta$, чтобы решение $y(a, y_0 ,\eta)$ в правом конце отрезка
совпало со значением из начальных условий. Другими словами,
решение исходной задачи эквивалентно нахождению корня уравнения

$$\Phi(\eta) = 0$$

где $\Phi(\eta) = \alpha_1 y + \alpha_2 y' - \beta$,
$\alpha_1$, $\alpha_2$, $\beta$ - коэффициенты уравнения правой границы.

Следующее значение искомого корня определяется по соотношению

$$\eta_{j+2} = \eta_{j+1} - \frac{\eta_{j+1}-\eta_j}
{\Phi(\eta_{j+1}) - \Phi(\eta_j)}\Phi(\eta_{j+1})$$

\subsection*{Конечно-разностный метод}

Рассмотрим двухточечную краевую задачу для линейного дифференциального
уравнения второго порядка на отрезке $[a,b]$:

$$\begin{aligned}
    & y'' + p(x)y' + q(x)y = f(x) \\
    & y'(a) = z_0 \\
    & \alpha_1 y(b) + \alpha_2 y'(b) = \beta \\
\end{aligned}$$

Введем разностную аппроксимацию производных следующим образом

$$\begin{aligned}
    & y'_k = \frac{y_{k+1} - y_{k-1}}{2h} + O(h^2) \\
    & y''_k = \frac{y_{k+1} - 2y_k + y_{k-1}}{h^2} + O(h^2) \\ 
\end{aligned}$$

Подставляя аппроксимации, приводя подобные и учитывая граничные
условия, получим систему линейных алгебраических
уравнений с трехдиагональной матрицей коэффициентов

$$\begin{cases}
    &-y_1 + y_2 = hz_0 \\
    &\left(1 - \dfrac{hp(x_k)}{2}\right)y_{k-1} + (h^2q(x_k) - 2)y_k
        + \left(1 + \dfrac{hp(x_k)}{2}\right
        
        )y_{k+1} = h^2f(x_k),
        \qquad k = 2,\dots,N-2 \\
    &-\alpha_1 y_{N-1} + (h\alpha_2 + \alpha_1)y_N = h\beta \\
\end{cases}$$

\pagebreak