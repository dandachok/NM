\section*{Реализация}

\subsection*{Метод стрельбы}

\begin{lstlisting}[language=C++]
class ShootingMethodL2R3 {
    public:
        ShootingMethodL2R3(BVProblemL2R3 task) : task(task), f(task.f) {}

        void Calc (double h, double eps) {
            x = GetValInRange(task.left, task.right, h);
            CauchyProblem ctask1 = task;
            CauchyProblem ctask2 = task;
            ctask1.y0 = 1;
            ctask2.y0 = 0;
            RungeKutta2 rg1(ctask1);
            RungeKutta2 rg2(ctask2);
            double& n1 = rg1.y0;
            double& n2 = rg2.y0;
            vdouble y1 = rg1(h);
            vdouble y2 = rg2(h);
            if (isFinish(y1, eps)) {
                y = y1;
                return;
            } else if (isFinish(y2, eps)) {
                y = y2;
                return;
            }
            do {
                double next_n = GetNextN(n1, n2, y1, y2);
                n1 = n2;
                n2 = next_n;
                y1 = y2;
                y2 = rg2(h);
            } while (!isFinish(y2, eps));

            y = y2;
        }

        vdouble GetY() const {
            return y;
        }

    private:

        double GetYDer(const vdouble& y, double xp) {
            int i = 0;
            for (; i < x.size() - 2; ++i) {
                if (x[i] <= xp && xp <= x[i + 1]) {
                    break;
                }
            }
            return (y[i + 1] - y[i]) / (x[i + 1] - x[i]);
        }

        double GetPhi (const vdouble& y) {
            double y_der = GetYDer(y, task.right);
            double phi = task.re.b * y_der + task.re.a * y[y.size() - 1] - task.re.c;
            return phi;
        }

        double GetNextN(double n1, double n2, const vdouble& y1, const vdouble& y2) {
            double phi1 = GetPhi(y1);
            double phi2 = GetPhi(y2);

            return n2 - (n2 - n1) / (phi2 - phi1) * phi2;
        }

        bool isFinish(const vdouble& y, double eps) {
            double phi = GetPhi(y);
            return std::abs(phi) < eps;
        }

        Function3* f;
        BVProblemL2R3 task;

        vdouble x;
        vdouble y;
};
\end{lstlisting}

\subsection*{Конечно-разностный метод}

\begin{lstlisting}[language=C++]
class FDMethod {
    public:
    FDMethod(Function1* p, Function1* q, BVProblemL2R3 t) :
        p(p), q(q), task(t) {}

    void Calc(double h) {
        x = GetValInRange(task.left, task.right, h);
        int n = (task.right - task.left) / h;
        vvdouble mat(n + 1, vdouble(4));
        mat[0][0] = 0;
        mat[0][1] = -1;
        mat[0][2] = 1;
        mat[0][3] = task.z0 * h;

        for (int i = 1; i < n; ++i) {
            mat[i][0] = 1 - (*p)(x[i]) * h / 2;
            mat[i][1] = (*q)(x[i]) * h * h - 2;
            mat[i][2] = 1 + (*p)(x[i]) * h / 2;
            mat[i][3] = 0;
        }

        mat[n][0] = -task.re.b;
        mat[n][1] = h * task.re.a + task.re.b;
        mat[n][2] = 0;
        mat[n][3] = task.re.c * h;

        y = SweepMethod(mat);
    }

    vdouble GetY() {
        return y;
    }

    private:
        Function1* p;
        Function1* q;
        BVProblemL2R3 task;

        vdouble x;
        vdouble y;
};
\end{lstlisting}

\pagebreak