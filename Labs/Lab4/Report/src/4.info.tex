\section*{Теория}

Метод вращений Якоби применим только для симметрических
матриц $A_{nxn}$ $(A = A^T)$ и решает полную проблему собственных
значений и собственных векторов
таких матриц. Он основан на отыскании с помощью итерационных
процедур матрицы $U$ в преобразовании подобия $\Lambda = U^{-1}AU$ , а
поскольку для симметрических матриц $A$
матрица преобразования подобия $U$ является ортогональной 
$(U^{-1} = U^T)$, то $\Lambda = U^T AU$, где $\Lambda$ - диагональная 
матрица с собственными значениями на главной диагонали

Пусть дана симметрическая матрица $A$. Требуется для нее вычислить с
точностью $\varepsilon$ все собственные значения и соответствующие им собственные
векторы. Алгоритм метода вращения следующий:

Пусть известна матрица $A^{(k)}$ на $k$–й итерации, при этом для $k=0$ $A^{(0)} = A$.
\begin{enumerate}
    \item Выбирается максимальный по модулю недиагональный элемент $a(k)$ 
    матрицы
    $$A^{(k)}\left(|a^{(k)}_{ij}| = \max_{l<m} |a^{(k)}_{lm}|\right)$$
    \item Ставится задача найти такую ортогональную матрицу $U^{(k)}$,
        чтобы в результате преобразования подобия 
        $A^{(k+1)} =U^{(k)T} A^{(k)}U^{(k)}$ произошло
        обнуление элемента $a^{(k+1)}_{ij}$ матрицы $A^{(k+1)}$.
    \item Строится матрица $A^{(k+1)}$
    
        $A^{(k+1)}=U^{(k)T}A^{(k)}U^{(k)}$,
    где $a^{(k+1)}_{ij} \approx 0$.

    В качестве критерия окончания итерационного процесса используется
    условие малости суммы квадратов внедиагональных элементов:
    $$t\left(A^{(k+1)}\right)= \sqrt{\left(\sum_{l,m;l<m} \left(a^{(k+1)}_{lm} \right)^{2}\right)}$$.
    
    \pagebreak
    Если $t\left(A^{(k+1)}\right) > \varepsilon$, то итерационный процесс
    $$A^{(k)} =U^{(k)T} A^{(k)} U^{(k)} =U^{(k)T} U^{(k-1)T} \dots U^{(0)} A^{(0)} U^{(0)} U^{(1)} \dots U^{(k)}$$
    продолжается. Если $t\left(A^{(k+1)}\right) < \varepsilon$ , то итерационный процесс останавливается, и в качестве
    искомых собственных значений принимаются 
    $\lambda_1 \approx a_{11} , \lambda_2 \approx a_{22} ,
        \dots,\lambda_n \approx a_{nn}$.
    Координатными столбцами собственных векторов матрицы $A$ в 
    единичном базисе будут столбцы матрицы 
    $U^{(1)} = U^{(0)} U^{(1)} ...U^{(k)}$, т.е.
    $$x^1 = \begin{pmatrix}
        u_{11} \\
        u_{21} \\
        \vdots \\
        u_{n1}
    \end{pmatrix},\quad
    x^2 = \begin{pmatrix}
        u_{12} \\
        u_{22} \\
        \vdots \\
        u_{n2}
    \end{pmatrix},\quad
    \dots,\quad
    x^n = \begin{pmatrix}
        u_{1n} \\
        u_{2n} \\
        \vdots \\
        u_{nn}
    \end{pmatrix}$$
    причем эти собственные векторы будут ортогональны между собой, т.е.
    
    $(x_l,x_m) \approx 0$, $l \neq m$.
\end{enumerate}

\pagebreak